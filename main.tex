\documentclass[12pt]{article}
\usepackage[a4paper]{geometry}
\usepackage[myheadings]{fullpage}
\usepackage{fancyhdr}
\usepackage{lastpage}
\usepackage{graphicx, wrapfig, subcaption, setspace, booktabs}
\usepackage[T1]{fontenc}
\usepackage[font=small, labelfont=bf]{caption}
\usepackage{fourier}
\usepackage[protrusion=true, expansion=true]{microtype}
\usepackage[spanish]{babel}
\usepackage{sectsty}
\usepackage{url, lipsum}
\usepackage{graphicx}
\usepackage[utf8]{inputenc}
\usepackage{courier}
\usepackage{ulem}
\usepackage{spverbatim}
\usepackage{multirow} %para las tablas
\usepackage{color}
\usepackage{graphicx}
\usepackage{epsfig}
\usepackage{multirow}
\usepackage{colortbl}
\usepackage{xcolor}
\usepackage{float}

\usepackage{adjustbox}
\usepackage{amsmath}

\usepackage{listings} %For code in appendix
\lstset
{ %Formatting for code in appendix
    language=Matlab,
    basicstyle=\footnotesize,
    numbers=left,
    stepnumber=1,
    showstringspaces=false,
    tabsize=4,
    breaklines=true,
    breakatwhitespace=false,
}

\newcommand{\HRule}[1]{\rule{\linewidth}{#1}}
\onehalfspacing
\setcounter{tocdepth}{5}
\setcounter{secnumdepth}{5}

%-------------------------------------------------------------------------------
% HEADER & FOOTER
%-------------------------------------------------------------------------------
\pagestyle{fancy}
\fancyhf{}
\setlength\headheight{15pt}
\fancyhead[R]{PROGRAMACIÓN}
\fancyfoot[R]{\thepage}
%-------------------------------------------------------------------------------
% TITLE PAGE
%-------------------------------------------------------------------------------

\begin{document}

\title{ \normalsize \textsc{PROGRAMACIÓN}
		\\ [2.0cm]
		\HRule{2pt} \\
		\LARGE \textbf{\uppercase{Proiecto Proyectos}}
		\HRule{2pt} \\ [0.5cm] \bigskip
		\normalsize \today \vspace*{1\baselineskip}}

\date{}

\author{
        \emph{Autores:} \\ \\
		Xabier Garrote }

\maketitle
\newpage
\tableofcontents
\newpage


%--------------------------------------------------------------------% Las partes del documento de aqui en adelante ----------------------------------------------------------------------
\section{Objetivo del programa}
\section{Menu principal}
\section{Menu clientes}
\section{Menu servicios}
\section{Manual de usuario del programa}
El programa se compone de las siguientes partes:


\end{document}